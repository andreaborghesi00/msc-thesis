\chapter{Conclusions}
% Summary of the work, and discuss how the research question and objectives were addressed in this work. 

This thesis undertook an investigation into the development of a computationally efficient, 2D-based pipeline for the detection and classification of pulmonary nodules in CT scans, with a significant focus on adapting explainability methods for the complex task of object detection. This concluding chapter summarizes the key contributions of the work, revisits the research questions posed in the introduction, discusses the principal findings and their implications, acknowledges the study's limitations, and proposes directions for future research.

\section{Summary of Contributions and Research Questions}
\label{sec:summary_contributions}

The primary motivation for this work was to address the challenges of high computational costs and the lack of transparency in AI models for medical imaging. The research was guided by two central questions, which this thesis has successfully addressed.

\paragraph{RQ1: How can 2D object detection approaches achieve clinically relevant performances for lung nodule detection within computational constraints?}
This question was addressed through the systematic development and evaluation of a complete 2.5D detection pipeline. The results demonstrate that by combining a lightweight Faster R-CNN architecture with an efficient backbone like EfficientNetV2s, it is possible to achieve a strong performance baseline. Key methodological contributions that enabled this include:
\begin{itemize}
    \item The successful implementation of a 2.5D input representation, which incorporates local volumetric context into a 2D framework, significantly improving performance over single-slice approaches.
    \item The development of an Informed HU Thresholding strategy for dataset pruning, which was shown via an ablation study to effectively reduce annotation noise and improve model performance.
\end{itemize}
While the final performance is not yet at a level for autonomous clinical deployment, this work establishes the viability of lightweight, 2D-based methods as a computationally accessible alternative to resource-intensive 3D models.

\paragraph{RQ2: How can explainability techniques be adapted for the structured outputs of object detection models?}
This question was addressed by focusing on gradient-free CAM techniques to circumvent the non-differentiable operations in the Faster R-CNN pipeline. The core contribution was the development of the \texttt{FasterRCNNBoxScoreTarget} function, a novel similarity metric designed to replace the mathematically undefined subtraction operation in the Score-CAM and SS-CAM algorithms. This adaptation provides a robust and principled methodology for applying these powerful, class-discriminative XAI techniques to object detectors.\\


The subsequent quantitative evaluation, conducted using the Inverse Distance Game, yielded a clear result. The experiments revealed that Eigen-CAM produced the most faithful explanations, achieving a mean score of 0.900, significantly higher than the 0.785 and 0.792 scores of both Score-CAM and SS-CAM as proved by the t-paired test. Furthermore, this superior localization performance was achieved with orders of magnitude greater computational efficiency, making Eigen-CAM the only method suitable for practical, near real-time applications. This finding suggests that for a well-trained, single-class detector, the class-agnostic simplicity of Eigen-CAM is more effective and vastly more practical than the more complex, adapted class-discriminative methods.

\section{Limitations of the Study}
\label{sec:limitations}

The primary limitation is the inherent trade-off of the 2.5D approach; while efficient, it does not capture the full, long-range volumetric context that a true 3D model could leverage. Secondly, the models were trained and evaluated primarily on the DLCSD24 dataset, and their generalization performance on data from different institutions or CT scanners remains to be tested. Finally, as discussed in Chapter 5, the performance of the classification module, while establishing a valuable baseline, requires further improvement before it could be considered clinically robust.

\section{Future Directions}
\label{sec:future_directions}

The findings and limitations of this thesis lay the groundwork for several promising avenues of future work:
\begin{itemize}
    \item \textbf{Enhancing the Detection and Classification Models:} Future efforts should focus on exploring more advanced lightweight architectures, potentially lightweight 3D CNNs or hybrid 2D/3D models, to bridge the performance gap with larger models while maintaining efficiency.

    \item \textbf{End-to-End Multi-Task Learning:} The current pipeline is sequential. A future direction could be to develop an end-to-end model that performs both detection and classification simultaneously, potentially allowing features to be shared and learned more effectively.

    \item \textbf{Training on multiple medical dataset} To ensure clinical relevance, the pipeline should be trained on multiple clinical dataset of relevance, ensuring proper model generalization.
    
    \item \textbf{Broader Exploration of XAI for Object Detection:} The adapted CAM methods should be compared against other families of XAI techniques specifically designed for object detection, such as those for attention-based models (e.g., DETR), to further advance the field of explainable medical object detection.
\end{itemize}

\subsection{Concluding Remarks}
\label{sec:concluding_remarks}

In conclusion, this thesis successfully designed, implemented, and validated a lightweight, 2D-based pipeline for lung nodule detection and classification. It provides a robust framework for adapting and quantitatively evaluating gradient-free explainability methods for object detection models. The work serves as a comprehensive proof-of-concept, establishing performance baselines and offering a clear and promising path for future research toward the development of efficient, effective, and trustworthy AI tools for medical imaging.

% \section{Future Directions}
% Potential future directions, such as exploring lightweight attention based methods, segmentation-based approaches, and comparing the adapted CAM methods with other explainability techniques ad-hoc designed for object detection tasks.
