\chapter{Introduction}
\label{ch:introduction}

\section{Problem Statement and Motivation}

Lung cancer remains one of the leading causes of cancer-related mortality worldwide, with early detection being crucial for improving patient outcomes and survival rates. Computed Tomography (CT) screening programs, such as the National Lung Screening Trial (NLST), have demonstrated the potential to reduce lung cancer mortality through early identification of pulmonary nodules \cite{aberle2011reduced}, \cite{de2020reduced}. However, the manual interpretation of CT scans is a time-intensive process that places significant burden on radiologists, while also being subject to inter-observer variability and potential oversight of small or subtle lesions.

Radiologists are often required to review hundreds of slices per patient, sometimes across dozens of patients in a single day. Under such conditions, cognitive fatigue can accumulate, potentially leading to decreased diagnostic precision, delayed reading times, or even missed findings, especially for low-contrast or small nodules that may appear on only a few slices \cite{stec2018systematic, taylor2019fatigue}. This challenge is further amplified in high-volume screening programs, where maintaining consistent accuracy over long hours is difficult even for experienced professionals.

Artificial intelligence (AI) based detection tools have the potential to alleviate this strain by automatically flagging suspicious regions of interest, enabling radiologists to focus attention on the most relevant slices. Such systems do not replace human expertise but can act as a second reader, improving sensitivity to subtle findings, reducing oversight caused by fatigue, and providing explainable visual cues that support decision-making and increase trust in automated recommendations \cite{glikson2020human}. 
On this note, it is important to mention that the use of AI in radiology is not always well-accepted, as it has been shown by Liu et al. \cite{liu2024artificial} that radiologists with a higher workload and lower AI-acceptance are more likely to experience burnout. Regardless, the integration of AI tools into clinical workflows has been shown to enhance diagnostic accuracy  and efficiency, ultimately leading to better patient care and outcomes \cite{guermazi2022improving,huynh2020artificial}

Nevertheless, existing AI solutions are often limited by computational demands and lack transparency in their predictions, motivating the need for efficient and explainable detection methods specifically tailored to lung nodule identification in CT scans.
This need is further reinforced by the recent European Union Artificial Intelligence Act, which establishes a regulatory framework that emphasizes transparency, accountability, and human oversight for AI systems—particularly in high-risk domains such as healthcare \cite{eu_ai_act}.

\section{Research Challenges}


\section{Research Questions and Objectives}

\section{Thesis Contributions}

\section{Thesis Structure}