\documentclass{book} 
% \usepackage{classicthesis}
\usepackage{thesis}
% \setstretch{1.05} % 1.1 line spacing

\raggedbottom
\begin{document}
\fontsize{11.5pt}{14pt}\selectfont
 
\includepdf{thesistitlepage/titlepage.pdf}
\cleardoublepage
\pagenumbering{gobble}
\chapter*{Abstract}
    The early detection of pulmonary nodules through Computed Tomography (CT) screening is crucial for improving patient outcomes in lung cancer, yet the manual review process is a significant burden on radiologists. While Deep Learning models offer a promising solution, state-of-the-art 3D approaches are often computationally expensive and lack transparency, hindering their widespread adoption.

    This thesis addresses these challenges by developing and evaluating a lightweight, 2D-based pipeline for the detection and explainability of lung nodules in CT scans. The core of the pipeline is a Faster R-CNN architecture, systematically evaluated with various efficient backbones. To mitigate the loss of volumetric information inherent in a 2D approach, a 2.5D input representation that stacks adjacent slices is employed. A secondary classification model is also developed to characterize detected nodules as benign or malignant.

    A central contribution of this work lies in adapting and quantitatively evaluating gradient-free Class Activation Map (CAM) methods for the task of object detection. A custom similarity metric was designed to make Score-CAM and Smoothed Score-CAM (SS-CAM) compatible with the structured outputs of the detector. The faithfulness of these methods, alongside Eigen-CAM, was objectively measured using a novel ``Inverse Distance Game" protocol.

    The results demonstrate that the lightweight detection pipeline establishes a strong performance baseline, with the EfficientNetV2-S backbone proving most effective. For explainability, the quantitative evaluation reveals that Eigen-CAM provides significantly more faithful and well-localized explanations than the adapted Score-CAM and SS-CAM variants, while being orders of magnitude more computationally efficient.

    Ultimately, this research validates the feasibility of computationally accessible 2D methods for lung nodile detection and contributes a framework for adapting and evaluating explainability techniques in object detection. The findings identify Eigen-CAM as a practical and effective method for providing trustworthy visual explanations in this medical imaging context.

\frontmatter
\cleardoublepage
\tableofcontents

\mainmatter

\chapter{Introduction}
\label{ch:introduction}

\section{Problem Statement and Motivation}

Lung cancer remains one of the leading causes of cancer-related mortality worldwide, with early detection being crucial for improving patient outcomes and survival rates. Computed Tomography (CT) screening programs, such as the National Lung Screening Trial (NLST), have demonstrated the potential to reduce lung cancer mortality through early identification of pulmonary nodules \cite{aberle2011reduced, de2020reduced}. However, the manual interpretation of CT scans is a time-intensive process that places significant burden on radiologists, while also being subject to inter-observer variability and potential oversight of small or subtle lesions.

Radiologists are often required to review hundreds of slices per patient, sometimes across dozens of patients in a single day. Under such conditions, cognitive fatigue can accumulate, potentially leading to decreased diagnostic precision, delayed reading times, or even missed findings, especially for low-contrast or small nodules that may appear on only a few slices \cite{stec2018systematic, taylor2019fatigue}. This challenge is further amplified in high-volume screening programs, where maintaining consistent accuracy over long hours is difficult even for experienced professionals.

Artificial intelligence (AI) based detection tools have the potential to alleviate this strain by automatically flagging suspicious regions of interest, enabling radiologists to focus attention on the most relevant slices. Such systems do not replace human expertise but can act as a second reader, improving sensitivity to subtle findings, reducing oversight caused by fatigue, and providing explainable visual cues that support decision-making and increase trust in automated recommendations \cite{glikson2020human}. 
On this note, it is important to mention that it has been shown that the use of AI can instead increase fatigue to radiologists with higher workload if they have a low acceptance of AI \cite{liu2024artificial}. Regardless, the integration of AI tools into clinical workflows has been shown to enhance diagnostic accuracy  and efficiency, ultimately leading to better patient care and outcomes \cite{guermazi2022improving,huynh2020artificial}.

Nevertheless, existing AI solutions are often limited by computational demands and lack transparency in their predictions, motivating the need for efficient and explainable detection methods specifically tailored to lung nodule identification in CT scans.
This need is further reinforced by the recent European Union Artificial Intelligence Act, which establishes a regulatory framework that emphasizes transparency, accountability, and human oversight for AI systems—particularly in high-risk domains such as healthcare \cite{eu_ai_act}.

\section{Research Challenges}

Designing AI-based tools for lung nodule detection in CT scans involves navigating a complex landscape of technical and practical challenges that significantly limit the applicability of existing solutions in real-world clinical environments.

The most immediate challenge stems from the computational demands of state-of-the-art approaches. While three-dimensional convolutional neural networks can theoretically exploit the full volumetric context of CT scans to improve detection accuracy, their practical implementation reveals significant limitations. These models typically require substantial GPU memory, often exceeding 16GB, and demand extensive training times that can extend to days or weeks, even on high-end hardware such as A100 GPUs with 40GB of memory \cite{wu2018systematicanalysisstateoftheart3d}. While inference times for individual patient scans may be manageable, the development and iterative improvement of such models becomes prohibitively expensive and time-consuming. 
Such computational requirements create a fundamental mismatch with research development processes, where rapid experimentation and iteration are crucial for advancing the field, especially in small research groups with limited access to high-end hardware.
This computational constraint naturally leads to consideration of more efficient 2D approaches, which process individual CT slices rather than entire volumes. This approach significantly reduces memory requirements and training times, allowing for more agile development cycles, at the cost of losing correlations between slices that may be crucial for accurate nodule detection. Regardless, a 2D approach allows for better accessbility to healthcare professionals, as it can be run on standard laptops or workstations without the need for high-end GPUs.
Equally critical, yet often overlooked in the technical literature, is the challenge of explainability in object detection systems. The medical domain presents unique requirements for AI interpretability, driven by regulatory demands, clinical decision-making needs, and the fundamental requirement for physician trust and acceptance. However, most existing explainability methods were developed for image classification tasks \cite{chattopadhay_2018gradcam++,draelos2021hirescam,jiang2021layercam, selvaraju2019gradcam}, where the goal is to explain a single prediction score for an entire image. Object detection fundamentally differs in that as it produces structured outputs consisting of multiple bounding boxes, each with associated class probabilities and confidence scores.
This structural difference creates significant technical obstacles for applying standard explainability techniques.

\section{Research Questions and Objectives}
The challenges outlined in the previous section give rise to three primary research questions that guide this thesis:

\paragraph{RQ1: How can 2D object detection approaches achieve clinically relevant performance for lung nodule detection within computational constraints?}
This question addresses the fundamental trade-off between computational efficiency and detection performance. While 3D approaches theoretically offer superior performance by exploiting volumetric context, their computational demands make them impractical for many research and deployment scenarios. We investigate whether 2D slice-based detection can achieve acceptable performance levels while operating within the memory and processing constraints of standard hardware configurations.

\paragraph{RQ2: How can explainability techniques be adapted for the structured outputs of object detection models?}
Standard explainability techniques are designed for classification tasks with single prediction scores, but object detection produces complex structured outputs with multiple bounding boxes, confidence scores, and non-differentiable post-processing steps. This question explores how gradient-free Class Activation Map (CAM) methods can be modified to generate meaningful explanations for object detection predictions, addressing the technical challenges posed by structured outputs and non-differentiable operations.\bigskip

To address these research questions, this thesis pursues the following specific objectives:
\paragraph{O1: Develop a computationally efficient 2D object detection pipeline for lung nodule detection}
Implement and compare multiple object detection architectures and their variants.
Design preprocessing pipelines optimized for 2D slice-based analysis, including slice selection algorithms to maximize information content.
Achieve detection performance that demonstrates clinical viability while operating within 16GB GPU memory constraints.

\paragraph{O2: Adapt CAM methods for object detection tasks }

Research and adapt CAM methods that can handle structured object detection outputs.
Implement comparison algorithms that can assess similarity between structured predictions for explainability evaluation.
Implement end-to-end explainable object detection pipeline that provides both predictions and corresponding explanation maps.

\paragraph{O3: Demonstrate integration of detection and classification with explainability}

Extend the object detection pipeline with a classification head for detected regions.
Apply explainability methods to the detection components.

\paragraph{O4: Provide comprehensive experimental validation}
Evaluate the complete system on the established medical imaging datasets NLST \cite{nlst_data}, DLCSD24 \cite{dlcsd24, tushar2025ailunghealthbenchmarking}. 
% Compare performance against relevant baselines and alternative approaches
% Analyze computational efficiency and practical deployment considerations

\section{Thesis Structure}
\label{sec:thesis_structure}

This thesis is organized into six chapters, each building upon the last to present a comprehensive account of the research undertaken. The structure is designed to guide the reader from the foundational concepts to the final conclusions in a logical and coherent manner.

\paragraph{Chapter 2: Background} provides the necessary theoretical foundation for this work. It begins with a detailed overview of the fundamentals of object detection, discussing the distinctions between one-stage and two-stage detectors, as well as anchor-based and anchor-free methods. It then delves into the specifics of the RetinaNet and Faster R-CNN architectures, and formally defines the standard evaluation metrics of Average Precision and Average Recall. The second half of the chapter introduces the field of Explainable AI (XAI) for computer vision, focusing on Class Activation Maps (CAMs) and critically analyzing their inherent limitations when applied to object detection, thereby motivating the adaptations developed in this thesis.

\paragraph{Chapter 3: Data and Preprocessing} details the datasets and the extensive data preparation pipeline that form the empirical basis of this research. It introduces the two primary datasets used, the National Lung Screening Trial (NLST) and the Duke Lung Cancer Screening Dataset 2024 (DLCSD24). The chapter provides a step-by-step description of the preprocessing workflow, including the slice selection and pruning algorithms designed to handle annotation noise. This chapter also details the design of the various ablation studies, including those for dataset pruning and the 2.5D input representation. Finally, it describes the methodology for deriving the specialized dataset used for the subsequent nodule classification task.


\paragraph{Chapter 4: Methodology} presents the core experimental design of this research. It outlines the object detection and classification architectures, the choice of backbones, and the specific training procedures and hyperparameters. Furthermore, it provides a comprehensive description of the explainability methods, including the implementation details of the adapted CAM techniques and the design of the ``Inverse Distance Game" used for their quantitative evaluation.

\paragraph{Chapter 5: Results} presents the empirical findings of the experiments described in the previous chapter. This chapter is divided into three main parts: an analysis of the object detection performance; a quantitative evaluation of the explainability methods based on the Inverse Distance Game; and an analysis of the classification performance on the benign versus malignant nodule task.

\paragraph{Chapter 6: Conclusions} concludes the thesis by summarizing the key contributions and findings. It revisits the research questions posed in this introduction to discuss how they were addressed. The chapter also acknowledges the limitations of the study and, based on these, proposes several directions for future research.
\chapter{Background}
\label{ch:object-detection}
\section{Fundamentals of Object Detection}
% What's object detection? Describe the task itself, how it differs from  classification and other computer vision tasks. What are the usual approaches in the deep learning context (one-stage vs two-stage, anchor-based vs anchor-free, etc.)? What are the most common architectures and their characteristics? 
% highlight problems related to their structured outputs and the inherited challenges for gradient backpropagation 

Object detection is one of the fundamental task in computer vision that involves localizing and classifying multiple objects within an image. Unlike image classification and other tasks that assign a single label to an individual image, object detection requires predicting a set of bounding boxes, each associated with a class label and a confidence score. The challenging nature of this output stems from the variable cardinality of such sets, as the number of objects in an image can vary significantly, leading to complex structured prediction problems.

To address these challenges in a principled manner, it is useful to describe object detection as a structured prediction problem. This framework not only clarifies the task requirements but also provides a common language for comparing different detection architectures and loss functions.\\

Let an image be denoted as 
$
x \in \mathbb{R}^{H \times W \times C},
$
where $H$ and $W$ are the spatial dimensions and $C$ is the number of channels.
The output of an object detector is a finite set of detections:
$$
\mathcal{D} = \{ (b_i, y_i, s_i) \}_{i=1}^N,
$$
where:
\begin{itemize}
    \item $b_i = (x_i, y_i, w_i, h_i)$ represents the bounding box in pixel coordinates in COCO format (center position, width, height);
    \item $y_i \in \{1, \dots, K\}$ is the predicted class label, with $K$ the total number of object classes;
    \item $s_i \in [0,1]$ is the confidence score, often interpreted as the estimated probability that $b_i$ belongs to class $y_i$.
\end{itemize}

During training, ground-truth bounding boxes $b_i^\ast$ are typically transformed into target regression parameters relative to reference boxes (anchors or proposals) as:
$$
t_x = \frac{x^\ast - x_a}{w_a}, \quad
t_y = \frac{y^\ast - y_a}{h_a}, \quad
t_w = \log \frac{w^\ast}{w_a}, \quad
t_h = \log \frac{h^\ast}{h_a},
$$
where $(x_a, y_a, w_a, h_a)$ are the anchor box parameters, and $(x^\ast, y^\ast, w^\ast, h^\ast)$ are the corresponding ground-truth parameters.

The specific optimization objectives used for bounding box regression and classification are detailed in Section~\ref{sec:loss_functions}, following the description of the detection architectures.

The object detection task can thus be formalized as learning a function:
$$
f_\theta: \mathbb{R}^{H \times W \times C} \to \mathcal{P}(\mathbb{R}^4 \times \{1, \dots, K\} \times [0,1]),
$$
parameterized by $\theta$, that maps an input image to a set of bounding box--label--score triples.
The set-valued nature of $\mathcal{P}(\cdot)$ reflects that the number of detections varies between images.

This formalization underpins the taxonomy of detection architectures discussed in the following section, where we distinguish between one-stage and two-stage approaches, anchor-based and anchor-free formulations, and their respective training and inference paradigms.

\subsubsection{Feature Pyramid Networks (FPN)}
Before diving into the specific architectures, we introduce a widely used component in most object detection models: the Feature Pyramid Network (FPN).

Objects appear at widely varying scales, early CNN backbone layers produce high-resolution but semantically weak features (shallow layers), while deeper layers produce low-resolution but semantically strong features.
FPNs fuse these to obtain semantically strong, multi-scale feature maps at multiple resolutions, enabling detectors to handle small and large objects efficiently with a single backbone pass (as opposed to costly image pyramids)~\cite{lin2017fpn}.\\

Let $\{C_2,C_3,C_4,C_5\}$ be backbone feature maps with strides $\{4,8,16,32\}$ w.r.t.\ the input.
FPN builds $\{P_2,P_3,P_4,P_5\}$ by a top–down pathway and lateral merges as shown in Figure~\ref{fig:fpn-lateral}:

\begin{figure}[h]
    \centering
    \includegraphics[]{images/fpn-lateral.pdf}
    \caption{A building block illustrating the lateral connection and
    the top-down pathway, merged by addition \cite{lin2017fpn}}
    \label{fig:fpn-lateral}
\end{figure}


\subsection{One-Stage vs Two-Stage Detectors}

Object detectors can be broadly categorized into \emph{one-stage} and \emph{two-stage} paradigms. Both ultimately predict a variable-sized set of detections $\mathcal{D} = \{(b_i, y_i, s_i)\}$, but they differ in how they structure intermediate computations and where they allocate capacity for localization vs.\ classification.

\subsubsection{Two-stage detectors.}
Two-stage methods decompose detection into proposal generation followed by proposal classification and box refinement. Let $g_{\phi}$ denote a proposal mechanism (e.g., a region proposal network operating on a shared backbone), which maps an image $x$ to a set of candidate regions $\mathcal{P}$:
$$
\mathcal{P} = g_{\phi}(x) = \{p_j\}_{j=1}^{M}.
$$
A second function $f_{\theta}$ then classifies and refines these regions to produce final detections:
$$
\mathcal{D} = f_{\theta}(x, \mathcal{P}).
$$
Operationally, the pipeline is:
\begin{enumerate}
    \item Backbone (optionally with a feature pyramid) extracts multi-scale features from $x$.
    \item A proposal generator produces a sparse set of regions $\mathcal{P}$ (typically with non-maximum suppression to reduce redundancy).
    \item Per-region features are pooled/aligned from the shared feature maps and passed to a detection head to output class scores and refined boxes.
\end{enumerate}

The idea underlying two-stage detectors is to incentivise accuracy by first generating a manageable number of promising candidates, which can then be processed and refined with more complex per-region heads. This also allows for a more efficient use of computational resources, as the second stage can focus on a smaller set of regions rather than processing the entire image densely.
But unfortunately, this approach introduces a fair amount of complexity requiring careful design.

\subsubsection{One-stage detectors.}
One-stage methods simplify the detection pipeline by predicting detections directly from the shared feature maps without an explicit proposal stage. This design is motivated by the desire to maximize throughput and reduce latency, particularly in real-time applications. This type of detection is often referred to as \emph{dense predictions} because it involves predicting detections at every spatial location of the feature maps, often having to take into account hundreds of thousands of locations per image.
Conceptually, it implements a direct mapping
$$
\mathcal{D} = h_{\psi}(x),
$$
where $h_{\psi}$ produces, for each spatial location (and possibly for multiple predefined reference shapes), joint classification scores and box parameters. The pipeline is:
\begin{enumerate}
    \item Backbone (optionally with a feature pyramid) extracts multi-scale features from $x$.
    \item Lightweight prediction heads applied densely over feature maps output $(b, y, s)$ candidates.
    \item A single-stage post-processing (e.g., top-$k$ filtering and non-maximum suppression) yields $\mathcal{D}$.
\end{enumerate}
This design maximizes throughput and simplifies the graph, but operates under severe class imbalance, due to the fact that most locations tend to be related to a generic background class, and thus the model has to learn to distinguish between a very large number of background locations and a comparatively small number of foreground locations. This imbalance is often addressed with specialized loss functions, such as the Focal Loss \cite{lin2018focalloss}, which down-weights easy-to-classify examples and focuses training on hard negatives.
The precision of the predictions often relies on effective assignment and post-processing techniques.
An alternative way of addressing this imbalance, ironically, is to use a two-stage approach, where the first stage takes care of separating the foreground (potential candidates) from the background.

\paragraph{Comparative characteristics.}
One might ask "Which paradigm is better then?" and the answer would be, as usual, "it depends". One-stage detectors are generally faster and more resource-efficient, and that is because they were intended to address the latency and throughput requirements of real-time applications. This comes at the cost of some accuracy that two-stage detectors can achieve with their more careful filtering mechanisms and (usually) more complex per-region heads that come at the cost of increased compatuational requirements and latency. And similarly one-stage detectors, they were designed to be accurate rather than fast.

Ideally one would like to have the best of both worlds, and we can find in literature a number of approaches that tries to reach two-stage performance with one-stage efficiency, such as the RetinaNet and YOLO \cite{lin2018focalloss,redmon2016yolo}.

We can summarize the main differences between one-stage and two-stage detectors as follows:
\begin{itemize}
    \item \textbf{Computation.} Two-stage: cost scales with the number of proposals $M$ (feature pooling and per-RoI head). One-stage: cost scales with the number of feature locations (and reference shapes) processed densely.
    \item \textbf{Capacity allocation.} Two-stage allocates more parameters per candidate via the second-stage head; one-stage relies on shallow, shared heads and benefits from strong multi-scale features.
    \item \textbf{Class imbalance.} One-stage training observes extreme foreground/background imbalance due to dense supervision; two-stage partially mitigates this by filtering with proposals and sampling strategies.
    \item \textbf{Latency/throughput.} One-stage typically achieves lower latency and higher FPS; two-stage often offers higher accuracy at increased computational cost.
    \item \textbf{Post-processing.} Two-stage commonly applies suppression both after proposal generation and after final classification; one-stage applies it once on dense predictions (often per class).
\end{itemize}


% \paragraph{Outlook.}
% In what follows, we instantiate these paradigms with canonical representatives: a one-stage detector with dense predictions, and a two-stage detector with proposals followed by RoI-level classification and refinement, both using feature pyramids to extract multi-scale features. We subsequently discuss anchor-based vs.\ anchor-free formulations, which apply orthogonally to both paradigms.


\subsection{Anchor-Based vs.\ Anchor-Free}

Object detectors differ not only by staging (one- vs.\ two-stage) but also by how they parameterize candidate boxes. The two prevailing choices are \emph{anchor-based} (predefined reference boxes) and \emph{anchor-free} (direct box prediction without references). This design is largely orthogonal to the staging paradigm.

\subsubsection{Anchor-based detectors}
\label{sec:anchor_based_detectors}
Let $\mathcal{A}=\{a_k\}_{k=1}^{K}$ be a set of predefined anchors tiled over feature maps. 
Each anchor $a_k=(x_a,y_a,w_a,h_a)$ serves as a reference against which a detector predicts offsets $(t_x,t_y,t_w,t_h)$ to obtain a box:
$$
x = x_a + w_a t_x \quad
y = y_a + h_a t_y \quad
w = w_a e^{t_w} \quad
h = h_a e^{t_h}
$$

Anchors are generated at multiple scales and aspect ratios per spatial location and per FPN level $\ell$ (with stride $s_\ell$), e.g.\ scales $\{s_\ell \cdot \alpha\}$ and ratios $r \in \{1\!:\!2,\,1\!:\!1,\,2\!:\!1\}$.
The number of anchors generated for each image tends to be large, but since we're considering the spatial locations of the feature maps, and not the original image, this number is still manageable, although it can still be in the order of hundreds of thousands, depending on the number of feature maps, stride, number of scales, aspect ratios and size of the image.\\

Training uses \emph{IoU-based assignment}: for ground-truth boxes $\{b_i^\ast\}$, compute $M_{ik}=\mathrm{IoU}(b_i^\ast,a_k)$; for some given $\tau_{\text{pos}}, \tau_{\text{neg}} \in [0, 1]$ mark $a_k$ positive if $M_{ik}\ge \tau_{\text{pos}}$ for some $i$, negative if $M_{ik}\le \tau_{\text{neg}}$, and ignore otherwise. 

We now have the anchors partitioned into $\mathcal{A}^+$ (positive), $\mathcal{A}^-$ (negative), and $\mathcal{A}^0$ (ignored).
Positives receive a foreground class label $y^\ast$ and regression targets $t_k$; negatives receive the background label; ignored anchors are excluded from loss computation.
Training thus optimizes
$$
\mathcal{L}
= \frac{1}{N_{\text{cls}}}\sum_{k \in \mathcal{A}^+ \cup \mathcal{A}^-} \ell_{\text{cls}}(\hat{c}_k, c_k)
\;+\;
\lambda\,\frac{1}{N_{\text{reg}}}\sum_{k \in \mathcal{A}^+} \ell_{\text{reg}}(\hat{t}_k, t_k),
$$
where $\ell_{\text{cls}}$ is the classification/objectness loss and $\ell_{\text{reg}}$ (e.g., Smooth-$L_1$ or IoU-based) is applied \emph{only} to positives.
Negatives contribute \emph{only} to classification as background; ignored anchors contribute to neither term.

\paragraph{Suppressing redundancy.}
As seen in this section, anchor-based detectors generate a large number of anchors, many of which are near-duplicates (they overlap with the same ground-truth boxes). Ideally we would like only one anchor, hence one detection, per ground-truth, therefore we need a post-processing procedure to remove these redundant detections.

This is typically done in two steps: we first keep the top-$k$ detections per level/class, where $k$ is an hyperparameter (pre-NMS step), and then we apply Non-Maximum Suppression (NMS) to remove overlaps, producing the final set $\mathcal{D}$ \cite{neubeck2006efficient}.
Given a set of competing detections $\mathcal{D} = \{(b_i, y_i, s_i)\}$, NMS iteratively selects the detection with the highest score $s_i$ and removes all other detections that overlap with it above a threshold $\tau_{\text{nms}}$ (typically $0.5$). The process continues until no detections remain above the threshold.
As a result, NMS produces a final set of detections $\mathcal{D} = \{(b_i, y_i, s_i)\}$ that are non-overlapping and have the highest scores among the competing detections.
This procedure is clearly non-differentiable, and thus it is not possible to backpropagate through it, hence we always backpropagate on the pre-NMS detections.


\subsubsection{Anchor-free detectors.}
Anchor-free methods remove the predefined anchor set $\mathcal{A}$: each feature-map location $p$ is supervised directly.
A common approach is the one used in FCOS, which assigns each location $p=(x_p, y_p)$ to at most one ground-truth box $b^\ast$ if $p \in b^\ast$ (optionally restricted to a center region and a size range matched to level $\ell$).
After assignment, the location $p$ is labeled positive with class $y^\ast$ and regression targets given by distances to the four sides of $b^\ast$:
$$
\ell_p = x_p - x^\ast_{\text{left}},\quad
t_p = y_p - y^\ast_{\text{top}},\quad
r_p = x^\ast_{\text{right}} - x_p,\quad
b_p = y^\ast_{\text{bottom}} - y_p,
$$
decoded at inference as $[x_p-\ell_p,\; y_p-t_p,\; x_p+r_p,\; y_p+b_p]$.
Locations not assigned to any box act as background in the classification loss and do not contribute to regression.
Many designs also predict a per-location quality term (e.g.\ ``centerness'') to down-weight ambiguous border locations. 
Alternative anchor-free formulations include keypoint-based representations (e.g.\ object centers or corners) with local offsets~\cite{tian2019fcos,law2019cornernet,duan2019centernet}.\\

Anchor-free models avoid anchor tuning and can better handle unusual aspect ratios, but rely on effective assignment rules (inside-box, center sampling, size ranges) and robust post-processing.

\paragraph{Comparison and practical notes.}
\begin{itemize}
    \item \textbf{Design knobs} Anchor-based: choose scales/ratios, IoU thresholds, per-level priors. Anchor-free: choose assignment region, size ranges per level, and optional quality terms.
    \item \textbf{Label assignment} Anchor-based uses $\mathrm{IoU}$ with thresholds; anchor-free uses geometric inclusion at locations (often with center sampling). Advanced heuristics (e.g., ATSS/OTA) can be applied to either family \cite{zhang2020atss,ge2021ota}.
    \item \textbf{Computational profile} Similar dense heads; anchor-free can reduce memory and labels by removing large anchor sets.
\end{itemize}



\section{RetinaNet}
% a one-stage anchor-based architecture, essentially a toned-down Faster R-CNN, this is why i'd describe it first
% Describe the its architecture and components, address that it is born to address the class imbalance problem with the Focal Loss introduction, and its intended for medical imaging applications


RetinaNet is a one-stage, anchor-based detector that combines a backbone with a feature pyramid and two lightweight, densely-applied heads (classification and regression). Its key contribution is the \emph{focal loss}, which mitigates extreme foreground/background imbalance in dense prediction~\cite{lin2018focalloss}.

\begin{figure}[h]
    \centering
    \includegraphics[page=1,width=\linewidth]{images/retinanet.pdf}
    \caption{RetinaNet architecture overview \cite{lin2018focalloss}}
    \label{fig:retinanet}
\end{figure}


\subsubsection{Architecture}
A convolutional backbone (e.g., ResNet) feeds a feature pyramid $\{P_\ell\}$ (typically $P_3$--$P_7$). 
At each pyramid level $\ell$ with spatial size $H_\ell \times W_\ell$, and stride $s_\ell$, a set of $A$ anchors per location is tiled. 
Two subnetworks are applied at each level of the pyramid, producing per-anchor outputs:
\begin{itemize}
    \item \emph{Classification head:} a small CNN produces per-class logits $\hat{z}_{\ell,ij,a,c}$, using independent sigmoids (no softmax) over $c \in \{1,\dots,K\}$.
    \item \emph{Regression head:} a parallel CNN predicts box deltas $\hat{t}_{\ell,ij,a}=(\hat{t}_x,\hat{t}_y,\hat{t}_w,\hat{t}_h)$.
\end{itemize}

Once we obtain the structured outputs from the network, we apply the post-processing steps discussed 
in Section~\ref{sec:anchor_based_detectors} to discard redundant detections.

\paragraph{Losses.}
As previously mentioned, RetinaNet introduced the \emph{focal loss} to address the extreme class imbalance in dense predictions.
It essentially down-weights easy to classify examples and focuses training on hard negatives (Figure~\ref{fig:focal-loss}), which is particularly useful in medical imaging where the background class can dominate the training set. To do so it adds a modulating factor $(1-p_t)^\gamma$ to the cross-entropy loss, with a focus parameter $\gamma$. The focal loss is hence defined as follows:

$$
\mathrm{FL}(p_t) = -\,\alpha\,(1-p_t)^{\gamma}\,\log(p_t),
$$
with typical $\alpha{=}0.25$, $\gamma{=}2$.\\
\begin{figure}[h]
    \centering
    \includegraphics[width=0.55\linewidth]{images/focal-loss.pdf}
    \caption{Focal loss at different $\gamma$ values \cite{lin2018focalloss}. When $\gamma=0$, it is equivalent to cross-entropy loss. As $\gamma$ increases, the loss focuses more on hard examples, down-weighting easy ones.}
    \label{fig:focal-loss}
\end{figure}



Box regression is optimized for positives only, commonly with Smooth-$L_1$ on $(\hat{t}-t)$; IoU-family losses are also used in variants.


\section{Faster R-CNN}
two-stage anchor-based architecture, the most common one and the one that is used in most of the literature.
Describe its components, RPN, RoI pooling etc., it is the culmination of the R-CNN family of architecutres.

% \subsection{DETR}
% a one-stage anchor-free architecture, the first to introduce the transformer architecture in object detection, it is a fully end-to-end architecture that does not rely on anchors or region proposals, but rather directly predicts the bounding boxes and class labels in a single pass. Unfortunately, it is not suitable for our use case due to its large needs of data.

\section{Evaluation Metrics: Average Precision and Average Recall}
Explain the need to formally define evaluation metrics as COCO's definitions are shaky, seems like everyone's using them but no one ever explains them.

\section{Explainability in Deep Learning}
Quick introduction of explainability in deep learning, highlighting how models are essentially black boxes and the need to understand, although partially, their inner workings. This need is even more pronounced in the medical field, where explainability is a requirement for clinical acceptance and regulatory compliance. Highlight how explainability methods, especially the ones that we are going to cover, are desinged to provide \emph{insights} into the model's decision-making process, it does not make it fully explainable, but rather makes it a gray-box model, which is still better than a black box.
Trainsition to CAMs for computer vision tasks

\subsection{Class Activation Maps (CAM)}
What are CAMs, how they work, and their (usual) limitations. Describe GradCAM as a gradient-based method and then explain why it is problematic for object detection tasks due to their sensitivity to the structured outputs and the non-differentiable post-processing steps. 

\subsection{Adaptations for Object Detection}
Discuss gradient-free CAM methods ScoreCAM, SS-CAM and EigenCAM, highlighting how they can be adapted to work with object detection outputs. there's a bit of math involved here as we got to pinpoint what fails in their original implementations and how we can fix it.

\chapter{Data and Preprocessing}
\label{chap:data-and-preprocessing}
\section{Datasets: NLST and DLCSD24}
The development and evaluation of computer-aided detection systems for pulmonary nodules heavily rely on comprehensive and diverse datasets. In this work, we leverage two distinct datasets: the National Lung Screening Trial (NLST) \cite{nlst_data} and the Duke Lung Cancer Screening Dataset 2024 (DLCSD24) \cite{dlcsd24}. Each dataset presents unique several slices/volumes of CT scans, annotated with pulmonary nodule locations, necessary for training and validating our detection models.

Both datasets will serve as 2D datasets, meaning that each slice of the CT scan will be treated as a separate image. The slicing will be done along the axial plane, which is the most common orientation for viewing CT scans.
The NLST dataset already comes as a collection of 2D slices on the axial plane, while the DLCSD24 dataset comes as a collection of 3D volumes that will need to be sliced along our desired plane, and possibly only a subset of slices of interest -- namely those containing nodules -- will be retained for training and evaluation.

\subsection{National Lung Screening Trial (NLST)}
The NLST is a large-scale, randomized controlled trial conducted by the National Cancer Institute (NCI) to determine if spiral computed tomography (CT) screening could reduce lung cancer mortality compared to standard chest X-ray. It enrolled over 53,000 participants at 33 sites across the United States. For the purpose of this study, the NLST dataset provides a vast collection of low-dose CT scans, which are invaluable for training nodule detection models.
Unfortunately, only a small subset of the NLST dataset includes annotations for pulmonary nodules, and these only cover malignant nodules.
After this filtering, we are left with approximately 9000 scans, each containing on average one nodule, for a total of around 9000 nodules of size greater or equal to 4 mm.
Regardless the absence of benign annotations, the NLST can be effectively utilized to teach a model to recognize nodule-like structures. Furthermore, the NLST dataset can serve as an excellent source for pre-training detection models, allowing them to learn general features of pulmonary nodules before fine-tuning on datasets with more detailed annotations.


\subsection{Duke Lung Cancer Screening Dataset 2024 (DLCSD24)}
The Duke Lung Cancer Screening Dataset (DLCS 2024) is a large-scale, annotated collection of low-dose thoracic CT scans designed to support research in lung nodule detection and cancer risk assessment using modern CT technology \cite{dlcsd24}. The dataset was compiled from screening examinations conducted at the Duke University Health System between January 2015 and June 2021, and consists of 1,613 CT volumes drawn from a pool of 2,061 patients, with additional cases reserved for future releases. Within these scans, a total of 2,487 nodules were annotated through a semi-automated pipeline in which a detection model, pre-trained on the LUNA16 dataset, generated candidate nodules that were subsequently verified and refined using radiology reports and expert review. Manual adjustments were performed by a medical student and a fellowship-trained cardiothoracic radiologist, and independent spot checks confirmed that the resulting annotations achieved an accuracy exceeding ninety percent. The dataset is released in multiple parts, with versioned updates hosted on Zenodo to ensure accessibility and reproducibility, and represents the first publicly available, large-scale screening dataset acquired with up-to-date CT protocols. As such, DLCS 2024 provides a valuable benchmark for developing and evaluating artificial intelligence models for lung nodule detection and classification in contemporary clinical practice.
Compared to the NLST, the DLCSD24 dataset offers both malignant and benign nodule annotations, despite with a severe imbalance with a ratio of approximately 1:10. The nodules in this dataset are also generally smaller, a property that makes them more challenging to detect. The DLCSD24 dataset is therefore particularly well-suited for training and evaluating nodule detection models, as it provides a more comprehensive representation of the types of nodules that may be encountered in clinical practice.

\subsubsection{Slice Extraction from 3D Volumes}
The DLCSD24 dataset is provided as a collection of 3D CT volumes. As previously stated in this chapter, we want to extract a 2D dataset from these volumes by slicing them along the axial plane. This is done by iterating through each volume, and extracting each slice as a separate image. If we were to extract all slices, we would end up with an extremely large dataset, with the majority of the data being non-informative, as most slices do not contain any nodules. To mitigate this, we use the provided annotations to identify slices that contain nodules, and only retain those slices for our 2D dataset. This results in a more manageable dataset size, while still retaining the most relevant information for nodule detection. 

\begin{figure}[h]
    \centering
    \includegraphics[width=0.6\linewidth]{images/dlcs_sample_unprocessed.png}
    \caption{Example of an unprocessed slice extracted from the DLCSD24 dataset.}
    \label{fig:dlcs-sample-unprocessed}
\end{figure}

% The size of the resulting 2D dataset depends on the thickness of the slices in the original 3D volumes, but in order to have a consistent size across all scans, we resample all volumes to have a slice thickness of 1.25mm, height and width spacing of 0.7mm, and then extract all slices containing nodules, along with a few slices before and after each nodule-containing slice to provide some context. This results in a final dataset of approximately 10,000 slices, each containing at least one nodule.

% This resampling step is crucial, as it ensures that all slices have the same resolution and spacing, which allows models to have a consistent spatial understanding of the images. 


\section{Preprocessing Pipeline}
\label{sec:preprocessing}
% Resampling, and HU clipping, explaining why we're clipping and why those ranges and not others, this is supported by the HU table.
% We might include here also discarded preprocessing steps, such as outlier removal (although it was a post-process step), and lung masking through morphological operations.

The preprocessing pipeline is a crucial step in preparing the CT scan data for effective analysis and model training. This pipeline involves several steps, including resampling, normalization, and Hounsfield Unit (HU) clipping, each of which plays an significant role in enhancing the quality and consistency of the input data and therefore quality of the detection.

\subsubsection{Resampling}
\label{sec:resampling}
CT scans can vary significantly in terms of their spatial resolution and slice thickness (a voxel might represent a differently sized parallelepiped in millimeters depending on the scan configurations), which can introduce inconsistencies when training machine learning models. To address this, we resample all CT scans to a uniform voxel (pixels for the NLST dataset) size of 1.25mm in the axial direction and 0.7mm in the coronal and sagittal directions.
These sizes were chosen according to \cite{tushar2025ailunghealthbenchmarking} as they represent a good compromise between preserving anatomical detail and managing computational resources.
This resampling is performed using bilinear interpolation, which helps to maintain the integrity of the anatomical structures while ensuring that all scans have a consistent spatial resolution.
For the DLCSD24 dataset, this resampling is performed before the slice extraction step on the entire 3D volume. Such operation is performed usign the MONAI framework \cite{monai}.

As for the NLST dataset, since it is already provided as a collection of 2D slices, we only need to ensure that each slice has the correct in-plane resolution of 0.7mm. If any slice does not meet this requirement, it is resampled using bilinear interpolation to achieve the desired resolution and it has been performed using the SimpleITK library \cite{lowekamp2013simpleitk} to extract the slices metadata and resample using its built-in resampler. The output size, given the desired spacing, original size and current spacing, is computed as follows:
$$
\text{output\_size} = \left( \left\lfloor\frac{S_1 \cdot \delta_1}{\delta_1^\prime}\right\rceil, \left\lfloor\frac{S_2 \cdot \delta_2}{\delta_2^\prime}\right\rceil, 1 \right)
$$

Where \(S_i\) is the original size along dimension \(i\), \(\delta_i\) is the original spacing along dimension \(i\), and \(\delta_i^\prime\) is the desired spacing along dimension \(i\). The output size is rounded to the nearest integer to ensure that it represents a valid number of pixels. Since the NLST dataset is already 2D, the output size for the third dimension is set to 1 as a dummy value.

\subsubsection{Hounsfield Unit Clipping}
\label{sec:hu-clipping}
Hounsfield Units (HU) are a quantitative scale for describing radiodensity in medical CT imaging. Different tissues and materials in the body have characteristic HU values, which can be used to differentiate between them. Table~\ref{tab:hu-scale} summarizes the typical HU ranges for various substances commonly found in CT scans. For our purposes, we focus on the range from -1000 HU (representing air) up to +500 HU to include soft and hard tissues, while escluding extremely high values that could correspond to implants or artifacts.
\begin{table}[h!]
    \centering
    \begin{tabular}{|l|c|}
    \hline
    \textbf{Substance} & \textbf{HU Range} \\ \hline
    Air & $-1000$ \\ \hline
    Lung & $-700 \ \text{to} \ -600$ \\ \hline
    Fat & $-120 \ \text{to} \ -50$ \\ \hline
    Water & $0$ \\ \hline
    Cerebrospinal fluid & $+15$ \\ \hline
    Renal Parenchyma & $+30$ \\ \hline
    Blood & $+13 \ \text{to} \ +75$ \\ \hline
    Muscle & $+35 \ \text{to} \ +55$ \\ \hline
    Liver & $+40 \ \text{to} \ +60$ \\ \hline
    Soft tissue, IV Contrast & $+100 \ \text{to} \ +300$ \\ \hline
    Bone (Cancellous) & $+300 \ \text{to} \ +400$ \\ \hline
    Bone (Cortical) & $+500 \ \text{to} \ +1900$ \\ \hline
    Metal & $>+3000$ \\ \hline
    \end{tabular}
    \caption{Hounsfield Unit (HU) ranges of different substances.}
    \label{tab:hu-scale}
\end{table}




\chapter{Methodology}
\section{Object Detection Architectures and Backbones}
\label{sec:arch_and_backbones}
The core of our lung nodule detection pipeline is built upon a robust and well-established object detection framework. For this research, experiments will be conducted on the architectures discussed in the previous chapter: Faster R-CNN and RetinaNet.

While these architectures defines the overall detection process, the quality of the features extracted from the input image is can impact its success. This feature extraction is performed by a deep convolutional neural network, commonly referred to as the backbone. The choice of backbone directly influences the model's performance, computational complexity, and memory footprint. A central component of our methodology is therefore to systematically evaluate the impact of different backbone architectures on the lung nodule detection task.

To investigate the trade-off between model performance and computational efficiency, we conducted experiments with three distinct backbone architectures, each representing a different design philosophy:

\begin{itemize}
    \item \textbf{ResNet50:} A canonical and widely adopted architecture, ResNet50 serves as a robust and powerful baseline. Its use of residual connections allows for the effective training of very deep networks, making it a standard choice for a vast range of computer vision tasks. With 25.5 million parameters, it represents a high-performance, but computationally intensive, option.

    \item \textbf{MobileNetV2:} Designed specifically for efficiency and deployment on resource constrained devices, MobileNetV2 is a lightweight architecture with only 5.4 million parameters. It utilizes inverted residual blocks and linear bottlenecks to achieve a favorable balance between accuracy and computational cost, making it an ideal candidate for exploring the feasibility of more accessible clinical tools.

    \item \textbf{EfficientNetV2-S:} This model represents a modern approach to network design, seeking to optimally balance accuracy, training speed, and parameter efficiency. By systematically scaling network depth, width, and resolution, and employing novel architectural blocks, EfficientNetV2-S (with 21.4 million parameters) offers performance competitive with larger models while maintaining a smaller computational footprint.
\end{itemize}

In all experiments, each backbone was initialized with weights pre-trained on the ImageNet1k dataset. This transfer learning approach is a standard practice that leverages the rich, hierarchical features learned from a large-scale, general-purpose dataset to significantly improve performance and reduce training time on smaller, specialized datasets such as the one used in this study. 

\subsection{Loss functions}
\label{sec:loss_functions}
A central metric for bounding box quality is the \textit{Intersection over Union} (IoU):
$$
\mathrm{IoU}(b, b^\ast) = \frac{\mathrm{area}(b \cap b^\ast)}{\mathrm{area}(b \cup b^\ast)},
$$
which measures the degree of spatial overlap between a predicted box $b$ and a ground-truth box $b^\ast$.
Variants such as Generalized IoU (GIoU), Distance IoU (DIoU), and Complete IoU (CIoU) incorporate additional geometric terms to improve optimization stability, particularly when boxes do not overlap \cite{rezatofighi2019giou,zheng2019diou, zheng2021ciou}.


Bounding box regression is also commonly evaluated in coordinate space using Minkowski distances.
The general $p$-norm between two vectors $u,v \in \mathbb{R}^n$ is defined as:
$$
\| u - v \|_p = \left( \sum_{j=1}^n |u_j - v_j|^p \right)^{\frac{1}{p}}.
$$
Special cases include the Manhattan distance ($L_1$ norm, $p=1$):
$$
\| u - v \|_1 = \sum_{j=1}^n |u_j - v_j|,
$$
and the Euclidean distance ($L_2$ norm, $p=2$):
$$
\| u - v \|_2 = \sqrt{\sum_{j=1}^n (u_j - v_j)^2}.
$$

In practice, modern detectors often adopt the Smooth-$L_1$ loss \cite{girshick2015fastrcnn} for bounding box regression, which combines the robustness of $L_1$ for large errors with the stability of $L_2$ for small errors:
$$
\mathrm{SmoothL1}(x) =
\begin{cases}
0.5\,x^2, & \text{if } |x| < 1, \\
|x| - 0.5, & \text{otherwise}.
\end{cases}
$$
This formulation reduces sensitivity to outliers while maintaining differentiability at the origin, making it suitable for gradient-based optimization.

As for the classification component, Faster R-CNN employs the standard cross-entropy loss for binary classification tasks, while RetinaNet utilizes the Focal Loss \cite{lin2018focalloss} to address class imbalance by down-weighting easy negatives and focusing training on hard examples.

\section{Data Preprocessing}
\subsection{Train-time Augmentations}
\label{sec:augmentation}
To further enhance the robustness of our models and improve their generalization capabilities, we apply a series of data augmentations during training. These augmentations include random rotations, flips, and intensity variations, which help to simulate the variability encountered in real-world clinical settings. By exposing the model to a diverse set of augmented images, we aim to reduce overfitting and improve the model's ability to generalize to unseen data. These augmentations are applied on-the-fly during training, ensuring that the model sees a different version of the data in each epoch.

The specific augmentations applied are as follows:
\begin{itemize}
    \item \textbf{Horizonal Flips}: Each image has a 50\% chance of being flipped horizontally. This augmentation helps the model learn invariance to left-right orientation, which is particularly relevant in medical imaging where anatomical structures can appear in various orientations.
    \item \textbf{Shift, Scale and Rotate}: Each image has a 50\% change of being randomly shifted (up to 5\% of image dimensions), scaled (between 0.9 and 1.1 times the original size), and rotated (up to 15 degrees). These transformations simulate variations in patient positioning and imaging angles, enhancing the model's ability to recognize nodules under different conditions.
    \item \textbf{Random Brightness and Contrast}: Each image has a 50\% chance of having its brightness and contrast randomly adjusted. Brightness is modified by a factor between -0.1 and +0.1, while contrast is adjusted by a factor between 0.9 and 1.1. This augmentation accounts for differences in imaging protocols and equipment, helping the model to generalize across varying image qualities.
    \item \textbf{CLAHE (Contrast Limited Adaptive Histogram Equalization)}: \\Each image has a 50\% chance of being processed with CLAHE. This technique enhances local contrast and improves the visibility of structures in areas that are too dark or too bright, which is particularly useful in medical images where subtle features can be critical for diagnosis \cite{mishra2021clahe}.
\end{itemize}


\section{Object Detection Pipeline}
To synthesize the previously described components, this section outlines the complete, end-to-end workflow used for training and evaluating the lung nodule detection models. This systematic process ensures reproducibility and provides a clear framework for the experiments presented in this thesis. The workflow proceeds in three main stages:

\begin{enumerate}
    \item \textbf{Data Preparation.} The initial raw 3D CT volumes from the DLCS dataset are processed through a sequential pipeline to generate the final training data:
    \begin{itemize}
        \item Each CT volume is first resampled to a uniform voxel spacing (Section \ref{sec:resampling}).
        \item 2D axial slices are extracted from each 3D volumetric annotation.
        \item The resulting set of slices is filtered using the Informed HU Thresholding strategy to reduce annotation noise (Section \ref{sec:dataset_pruning}).
        \item Finally, the 2.5D three-channel input representation is constructed for each target slice (Section \ref{sec:2.5d_approach}).
    \end{itemize}

    \item \textbf{Model Training.} The Faster R-CNN/RetinaNet model is trained on the prepared dataset. 
    \begin{itemize}
        \item The impact of different backbone architectures (ResNet50, MobileNetV2, EfficientNetV2-S) is systematically evaluated (Section \ref{sec:arch_and_backbones}).
        \item The training process incorporates extensive online data augmentation (Section \ref{sec:augmentation}) to prevent overfitting and is guided by a consistent set of hyperparameters (AdamW optimizer, Cosine Annealing scheduler).
    \end{itemize}

    \item \textbf{Evaluation.} The trained models are evaluated on a held-out test set. 
    \begin{itemize}
        \item The models' predictions are compared against the ground truth annotations.
        \item Performance is quantitatively measured using the standard COCO metrics of Mean Average Precision (mAP) and Mean Average Recall (mAR), as formally defined in Section \ref{sec:eval_metrics}.
    \end{itemize}
\end{enumerate}


\section{Classification Head Extension}
Discuss the classification network, how we extracted a classification dataset from the detection dataset using the bounding boxes. 


\section{Explainability Methods}
Describe how are we using the adapted CAM methods, which layers we're targeting and an example of expected output.
Explain how we're going to compare the explainability methods using the segmentation game.
\chapter{Results}
\label{chap:results}
\section{Object Detection Performance Analysis}
%Present the results of the object detectors, making sure that each setup is clearly defined: architecture, backbone, training hyperparameters, eventual pretraining, etc.
%For each setup we present the performance using AP and AR metrics, their inference times, and the GPU memory usage.

Our experimental approach involved a two-stage strategy designed to address both the technical challenges of developing robust object detection models and the practical constraints of limited computational resources.

\subsection{Dataset Characgteristics and Usage Strategy}
The National Lung Screening Trial (NLST) dataset served as our initial development and pretraining dataset. NLST contains individual CT slices with annotations for malignant nodules, characterized by relatively large and visually evident nodules that are easier to detect. The individual slice format eliminated the need for complex 3D-to-2D conversion pipelines, allowing us to focus initially on core object detection methodology without the additional complexity of volume processing.
In contrast, the Duke Lung Cancer Screening (DLCS) dataset represents a more challenging and clinically realistic scenario. DLCS provides full CT volumes with annotations for both benign and malignant nodules, many of which are smaller and more subtle than those in NLST. The volumetric format required implementation of our complete preprocessing pipeline, including the slice selection and pruning algorithms developed to optimize 2D detection performance from 3D volumes discussed in Chapter \ref{chap:data-and-preprocessing}. This dataset was used for final model evaluation, providing a more rigorous test of our methods in a realistic clinical context.

To assess the optimal training approach for our challenging DLCS evaluation scenario, we designed an ablation study comparing two training strategies:
\begin{itemize}
    \item \textbf{NLST Pretraining}: Models were pretrained on the NLST dataset before being fine-tuned on the DLCS dataset. This approach leverages the larger, more visually distinct nodules in NLST to provide a strong initial feature representation.
    \item \textbf{Direct Training on DLCS}: Models were trained directly on the DLCS dataset from ImageNet-pretrained weights, without prior exposure to NLST. This strategy aims to adapt the model directly to the more challenging and clinically relevant nodules in DLCS.
\end{itemize}

This comparison allows us to evaluate whether the progressive difficulty approach provides significant benefits in terms of detection performance over direct training on the more complex DLCS dataset. 

The transfer learning strategy from NLST to DLCS tries to address the large domain gap between a CT dataset and ImageNet, where the latter is primarily focused on natural images. Nonetheless it is important to note that pretraining on ImageNet is often a standard practice in computer vision as it provides a strong initial feature representation, especially on the first layers of the network that tend to capture low-level features such as edges, textures, and basic shapes, all of which are easily transferable across different domains.

\subsection{Model Architectures and Training Details}
We mainly implemented two object detection architectures: Faster R-CNN and RetinaNet, both of which are widely used in the field. Each architecture was tested on several backbones, namely ResNet50, MobileNetV2, and EfficientNetV2S.
The experimental framework was built entirely from scratch, allowing us to customize each component of the object detection pipeline, including the backbone selection and anchor box sizes, along with the usual training hyperparameters such as learning rate and batch size.

The implementation of these two architectures was mainly provided by the \texttt{torchvision} library \cite{torchvision2016} . Unfortunately, it does not provide an implementation of these architectures with EfficientNetV2S as a backbone, so we had to implement it ourselves and attach it to an FPN head. The same applied for the MobileNetV2 backbone although for the RetinaNet architecture only. 

Each experiment used mixed precision training, which allows most computation to be performed using 16-bit floating-point (FP16) precision, while maintaining 32-bit floating-point (FP32) precision for critical operations that require numerical stability such as loss computation and gradient accumulation \cite{micikevicius2018mixedprecisiontraining}.
This approach significantly reduces GPU memory usage and accelerates training without compromising model accuracy. The implementation of mixed precision training was provided by the \texttt{torch.cuda.amp} module, which automatically manages the scaling of gradients and the conversion between FP16 and FP32 as needed. 

The following hyperparameters were kept constant for all experiments to ensure a fair comparison.
Due to the pretrained nature of every experiment, the learning rate was set to a low value of $0.0001$ to avoid catastrophic forgetting, and the batch size was set to $8$ to fit the GPU memory constraints. The training was performed for a maximum of $30$ epochs, with early stopping based on the validation metrics to prevent overfitting and cut down on training time with a patience of $10$ epochs. For the same reason the validation set was used to monitor the training process and decided to run a validation step every $3$ epochs.
The optimzer used was AdamW, which is a variant of the Adam optimizer that decouples weight decay from the optimization step, providing better generalization performance in many cases \cite{loshchilov2019decoupled, kingma2017adam}.

As for the scheduler used, we opted for a Cosine Annealing scheduler, which graduelly reduces the learning rate over the course of training, allowing for a more fine-tuned convergence towards the end of the training process. We also experimented with Cosine Annealing with Warm Restarts, which periodically resets the learning rate to a higher value, but it often led to failed training runs due to the mixed precision training induced instability, so we decided to stick with the simpler version.
Lastly, for the DLCS dataset, the 3-channels contain the previous, current, and next slice of the analyzed nodule, allowing the model to have a better context of the nodule's surroundings, effectively simulating a 3D input while still using a 2D object detection architecture.

The experiments were run using \texttt{parallel} to allow queuing multiple training runs on the same GPU, which is particularly useful for hyperparameter tuning and ablation studies. This approach allows us to efficiently utilize the available GPU and time resources.

\subsection{Experimental Results}
To quantitatively assess the performance of our models and the viability of the proposed training strategies, we evaluated each model configuration using the mean Average Precision (mAP) metric, a standard for object detection tasks. We report mAP at an Intersection over Union (IoU) threshold of 0.50 (mAP@.50), which is a common benchmark, as well as the average mAP over IoU thresholds from 0.50 to 0.95 in steps of 0.05 (mAP@.50:.95) to provide a more comprehensive assessment of localization accuracy. Additionally, we include mAP at a lower IoU of 0.10 (mAP@.10) to gauge the models' ability to detect nodules even with less precise bounding boxes.

\subsubsection{NLST Pretraining Results}
First, we present the results of the models trained on the NLST dataset, which served as a pretraining step before fine-tuning on the DLCS dataset.
As mentioned in the previous section, these models are pretrained on the ImageNet dataset, a training from scratch approach would have needed a much larger dataset to achieve comparable results, so we decided to use these weights as a starting point.
The results of the models trained on the NLST dataset are shown in Table \ref{tab:nlst-models}.

\begin{table}[h]
    \centering
    \begin{tabular}{lccc}
        \hline
        \textbf{Model} & \textbf{mAP@10} & \textbf{mAP@50} & \textbf{mAP@[5:95]} \\
        \hline
        \rowcolor{yellow!20}
        RetinaNet ENv2s      & 0.950 $\pm$ 0.015 & 0.756 $\pm$ 0.021 & 0.577 $\pm$ 0.028 \\
        RetinaNet MobileNet  & 0.935 $\pm$ 0.022 & 0.720 $\pm$ 0.029 & 0.560 $\pm$ 0.035 \\
        RetinaNet ResNet50   & 0.889 $\pm$ 0.019 & 0.645 $\pm$ 0.025 & 0.510 $\pm$ 0.031 \\
        FasterRCNN ENv2s     & 0.948 $\pm$ 0.014 & 0.742 $\pm$ 0.020 & 0.577 $\pm$ 0.027 \\
        FasterRCNN MobileNet & 0.908 $\pm$ 0.025 & 0.725 $\pm$ 0.027 & 0.559 $\pm$ 0.033 \\
        FasterRCNN ResNet50  & 0.891 $\pm$ 0.018 & 0.742 $\pm$ 0.019 & 0.546 $\pm$ 0.029 \\
        \hline
    \end{tabular}
    \caption{Object detection performance on the NLST dataset of Faster R-CNN and RetinaNet architectures with different backbones pretrained on COCO.}
    \label{tab:nlst-models}
\end{table}

Surprisingly, the RetinaNet architecture is comparable and, although not statistically significantly better, it presents slightly higher AP metrics compared to Faster R-CNN with the same backbone.
We will see later that this trend is reversed when the models are trained on the DLCS dataset, and from this one might infere that such a difference is due to the difference in difficulty of the datasets, with NLST being easier to detect, we might say that a simpler architecture is sufficient to achieve good results, and that the more complex Faster R-CNN is not able to extract more information from the data.

If we were to compare the results of the other backbones, we can see that the EfficientNetV2S backbone performs the best overall, with the MobileNet backbone beign the second best regardless of its extremely low number of parameters.

This results are also aiming to show that although many architectures are shipped with ResNet50 as a backbone, it is not the best choice for every task, especially when there are available more efficient and effective backbones that already exploit residual connections and other techniques to improve performance.

\subsubsection{DLCS Direct Training Results}
Next, we evaluated the models on the more clinically realistic and challenging DLCS dataset. The models were again initialized with ImageNet1K weights and trained directly on DLCS. This experiment was designed to quantify the difficulty of detecting the smaller, more subtle nodules characteristic of this dataset without the aid of domain-specific pretraining. The results are shown in Table \ref{tab:dlcs-models-not-pretrained}.

\begin{table}[h]
    \centering
    \begin{tabular}{lccc}
        \hline
        \textbf{Model} & \textbf{mAP@10} & \textbf{mAP@50} & \textbf{mAP@[5:95]} \\
        \hline
        RetinaNet ENv2s      & 0.671 & 0.427 & 0.363 \\
        RetinaNet MobileNet  & 0.736 & 0.362 & 0.352 \\
        RetinaNet ResNet50   & 0.678 & 0.424 & 0.364 \\
        \rowcolor{yellow!20} 
        FasterRCNN ENv2s     & 0.745 & 0.533 & 0.423 \\
        FasterRCNN MobileNet & 0.687 & 0.337 & 0.332\\
        FasterRCNN ResNet50  & 0.679 & 0.484 & 0.393 \\
        \hline
    \end{tabular}
    \caption{Object detection performance on the DLCS dataset of Faster R-CNN and RetinaNet architectures with different backbones pretrained on ImageNet1K.}
    \label{tab:dlcs-models-not-pretrained}
\end{table}

As expected, the performance of the models on the DLCS dataset is significantly lower than on the NLST dataset, reflecting the increased difficulty. As previously announced, now the Faster R-CNN architecture outperforms RetinaNet with EfficientNetV2S and ResNet50 backbones, while the MobileNet still performs better on with the RetinaNet architecture, making it a good candidate for lower-end devices.
The best performing model is the Faster R-CNN with EfficientNetV2S backbone, achieving an overall mAP@50:95 of 0.423, that is still not enough to be clinically useful, although its map@10 of 0.745 shows how it might be used as a screening tool, and that it performs its performance lowers quickly as the IoU threshold increases, which is a common issue with object detection models.

\subsubsection{DLCS Pretrained Results}
Finally, we evaluated the models pretrained on the NLST dataset and then fine-tuned on the DLCS dataset. This approach aims to leverage the larger, more visually distinct nodules in NLST to provide a strong initial feature representation for the more challenging nodules in DLCS, similar to a shallow curriculm learning approach.
The results of the models pretrained on NLST and then fine-tuned on DLCS are shown in Table \ref{tab:dlcs-models-pretrained}.

\begin{table}[h]
    \centering
    \begin{tabular}{lccc}
        \hline
        \textbf{Model} & \textbf{mAP@10} & \textbf{mAP@50} & \textbf{mAP@[5:95]} \\
        \hline
        RetinaNet ENv2s      & 0.665 & 0.404 & 0.343 \\
        RetinaNet MobileNet  & 0.675 & 0.245 & 0.297 \\
        RetinaNet ResNet50   & 0.664 & 0.412 & 0.350 \\
        FasterRCNN ENv2s     & 0.723 & 0.528 & 0.419 \\
        FasterRCNN MobileNet & 0.698 & 0.299 & 0.317 \\
        FasterRCNN ResNet50  & 0.693 & 0.497 & 0.386 \\
        \hline
    \end{tabular}
    \caption{Object detection performance on the DLCS dataset of Faster R-CNN and RetinaNet architectures with different backbones pretrained on NLST (see \ref{tab:nlst-models}).}
    \label{tab:dlcs-models-pretrained}
\end{table}

Unfortunately, the results of this fine-tuning approach are slightly worse than the direct training on DLCS with the ImageNet weights. Although for the best performing model this difference is not statistically significant, it is still a disappointing result.
This might be due to the fact that the NLST dataset is not large enough. Which is a common issue when fine-tuning on a small dataset, as the model might forget useful general features learned from the ImageNet dataset. Despite this problem can be mitigated by using a low learning rate and by freezing some of the layers -- both of which were done in this case -- it is yet a possibility when the dataset is small compared to its complexity. 
On a different note we can see that in this experimental setup the Faster R-CNN performs bettern than RetinaNet on every backbone, with the EfficientNetV2S backbone still being the best performing one.


\section{Explainability Evaluation}
To evaluate the explainability methods, we 

%Discuss the performance of the explainability methods, comparing them based on the segmentation game, performing some statistical analysis on the results.

\chapter{Conclusions}
Summary of the work, and discuss how the research question and objectives were addressed in this work. 

\section{Future Directions}
Potential future directions, such as exploring lightweight attention based methods, segmentation-based approaches, and comparing the adapted CAM methods with other explainability techniques ad-hoc designed for object detection tasks.

\bibliographystyle{plain}
\bibliography{bibliography/bibliography}

\include{src/acknowledgements}
\include{src/acknowledgements_ita}

\hyphenation{heat-maps}

\end{document} 
